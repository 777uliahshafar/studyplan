%%%%%%%%%%%%%%%%%%%%%%%%%%%%%%%%%%%%%%%%%
% Simple Article
% Integrated article template with simple for make4ht
% LaTeX Class
% Version 1.0 (10/11/20)
%
% This class originates by:
% Vel and  Nicolas Diaz
%
% Authors:
% Muhammad Uliah Shafar
%
%
% Free License:
%
%
%%%%%%%%%%%%%%%%%%%%%%%%%%%%%%%%%%%%%%%%%
\documentclass[12pt]{simart} % Font size (can be 10pt, 11pt or 12pt)

%----------------------------------------------------------------------------------------
%	TITLE SECTION
%----------------------------------------------------------------------------------------
% MAIN TITLE SECTION
\title{
\textbf{ Study Plan} \\
} % Title and subtitle
%\date{\textbf{\DTMtoday}}
\date{\textbf{\today}}
\author{Uliah}

%----------------------------------------------------------------------------------------
% OTHER TITLE SECTION

%\title{\textbf{Sistem Sarana dan Prasarana Jl. Pinggir Laut} \\ {\Large\itshape Infrastructure of Waterfront Parepare City}} % Title and subtitle

%\author{\textbf{Uliah Shafar} \\ \textit{Universitas Diponegoro}} % Author and institution

%\date{\today} % Date, use \date{} for no date

%----------------------------------------------------------------------------------------


\begin{document}

\maketitle
%----------------------------------------------------------------------------------------
%	ESSAY BODY
%----------------------------------------------------------------------------------------
\section{Prompt}

\section{language study plan :Study plans to improve Korean/English language ability* required for taking a degree course before and after you come to Korea.}

While I am still in my country, I would have planned my langugage study as followed:

March through June
\begin{itemize}
    \item I would get used to the korean conversation through electornic media platform such as youtube, netflix and viu.
    \item I would learn the basic of korean letter (Hangeul) from 90daykorean.com which is free.
    \item I would make a journal book of my journey as a beginner of korean language study.
    \item Memorize 10-15 words of Korean vocabulary each day.
    \item Finding and Participate in korean language community in my country, this might be online community.
    \item Study for IELTS exam to increase my score and develop my academic english skills through the official book and an internet (ielts-simon.study).
\end{itemize}

If I was elected as awarde of GKS in june, I would study more and exclusively through institution courses. By the end of my course, I would dabble on taking an exam of TOPIC in order to familiarize myself with the examination. After I completed all my preperation  upon the departure to Korea, I would have sufficient language to communicate in simple conversation, like in travel, grocery store and any place.

In my arrival in Korea, I would maximize my korea langugage study in one-year GKS's langugage program.
I will follow the study curriculum in the class and continue the study in dormniotry until I understand the class's objective of the day.
In addition, I will be working out our language with other peers in our spare time nonchalantly.
The program curriculum must succeed us since GKS program has pass more than a hundred thousand of applicant.

As the language program is running, I would add some addition activities. Participating in available organization for international student will have huge impact on my Korea improvement. For example, I would find and join volunteering activities surround campuss. And also try as much as i can surround myself with native korean speaker. I can interact with them maybe in dorminitory, park, or even outside libarary.

Even though someday I would have quite well with my korean daily conversation.


























\section{Goal of study \& Study Plan :Goal of study and detailed study plan}
\section*{Please describe your study plan for graduate study in your area of specialization. Be as specific as you can regarding your academic interests and the curriculum you expect to follow in achieving your goals.}

\section{Future Plan after Study :Future plan in Korea or another country after finishing GKS program}
\section*{Please describe your future plans or goals after graduate studies.}

In short term, I might continue living in South Korea for a while, if possible.  The chance of living in Korea would enable me to contribute more either in academic, professional, or social way. For example, I would find job opportunities there where I would work with korean and involve in some korean development. Those will give me further exposure of korean culture, education style, working style and even lifestyle that enhance my experience in korea.

In long term, I would like to continue delivering lecture in the university that has granted me further study consent. My lecture would have consisted of a mixed knowledge and experience from Korea and Indonesia. For example, I would like to present comparison of urban planning in the class spesifically in public space between the city in indonesia and korea. As result, my future student would not only understand its country architecture, but they would have wide perspective.

Beside delivering lecture, I would probably doing some research. My research topic would have discuss about urban planning topic in the developing city. For example, I would explore the strength and weakness public spaces on forming their identity between two country. The result of a number research of mine could be utilized by policy maker,planner, or even architect on figure out the problems by looking at two cases examples.

It is encouraged that lecturer do community services. I have planned to involve in community service program with student. The job of a lecture is to supervise students while doing their program. This program will benefit a place of program conducted, usually it will be in remote area.

%----------------------------------------------------------------------------------------
%	BIBLIOGRAPHY
%----------------------------------------------------------------------------------------

%\bibliographystyle{apalike}

%\bibliography{biblio.bib}



\end{document}

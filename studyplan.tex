%%%%%%%%%%%%%%%%%%%%%%%%%%%%%%%%%%%%%%%%%
% Simple Article
% Integrated article template with simple for make4ht
% LaTeX Class
% Version 1.0 (10/11/20)
%
% This class originates by:
% Vel and  Nicolas Diaz
%
% Authors:
% Muhammad Uliah Shafar
%
%
% Free License:
%
%
%%%%%%%%%%%%%%%%%%%%%%%%%%%%%%%%%%%%%%%%%
\documentclass[12pt]{simart} % Font size (can be 10pt, 11pt or 12pt)

%----------------------------------------------------------------------------------------
%	TITLE SECTION
%----------------------------------------------------------------------------------------
% MAIN TITLE SECTION
\title{
\textbf{ Study Plan} \\
} % Title and subtitle
%\date{\textbf{\DTMtoday}}
\date{\textbf{\today}}
\author{Uliah}

%----------------------------------------------------------------------------------------
% OTHER TITLE SECTION

%\title{\textbf{Sistem Sarana dan Prasarana Jl. Pinggir Laut} \\ {\Large\itshape Infrastructure of Waterfront Parepare City}} % Title and subtitle

%\author{\textbf{Uliah Shafar} \\ \textit{Universitas Diponegoro}} % Author and institution

%\date{\today} % Date, use \date{} for no date

%----------------------------------------------------------------------------------------


\begin{document}

\maketitle
%----------------------------------------------------------------------------------------
%	ESSAY BODY
%----------------------------------------------------------------------------------------
\section{Prompt}

\section{language study plan :Study plans to improve Korean/English language ability* required for taking a degree course before and after you come to Korea.}

While I am still in my country, I would plan my Korean language study as follows:

In March to June
\begin{itemize}
    \item I would get used to the Korean conversation through electronic media platforms such as Youtube, Netflix and Viu.
    \item I would learn the basic of Korean letter (Hangeul) from 90daykorean.com which is free.
    \item I would make journal books of my journey as a beginner of Korean language study.
    \item Memorize 10-15 words of the Korean vocabulary each day.
    \item I would find and participate in Korean language communities in my country, this might be online communities.
    \item Study for the IELTS exam to increase my score and develop my academic English skills through the official book and an internet (ielts-simon.study).
\end{itemize}

If I was elected as an awardee of the Global Korea Scholarship in June, I would study more and exclusively through an institution course. By the end of my course, I would dabble on taking an exam of TOPIC in order to familiarize myself with the examination. After I completing all the preparation upon the departure to Korea, I would have sufficient language to communicate in a simple conversation, like in a travel, a grocery store and any place.

On my arrival in Korea, I would maximize my Korean language study in a one-year GKS's language program.
I will follow the study curriculum in the classes and continue the study in my dormitory until I understand the class's objective of the day.
In addition, I will be working a Korean language out with other peers in our spare time casually.
The program curriculum must succeed us since GKS program has pass more than a hundred thousand of applicant.

As the language program is running, I would add some addition activities. An extra activity would have a huge impact on my language improvement such as participating in available organization for international students.
For example, I would find and join volunteering activities a campus surrounding. And also try as much as I can surround myself with native Korean speakers. I can interact with them maybe in a dormitory, a park, or even outside a library.

Even though someday I would fluent on a daily conversation of a Korean language, my language grade can only be examined through the topic test. It is always available to do a test each semester. If I feel well-prepared, I will take an exam until I get the deserved score. This score would certainly give me a confidence on pursuit a doctorate degree in the Korean University.

The architecture department programs in most of Korean universities use an English as a language program. Therefore, after passing on the TOPIK exam, I would catch up on an academic English study. I would try to take one more IELTS test in early doctorate program. While I catch up with my English, I will always maintain my Korean language throughout the studies.


\section{Goal of study \& Study Plan :Goal of study and detailed study plan}
\section*{Please describe your study plan for graduate study in your area of specialization. Be as specific as you can regarding your academic interests and the curriculum you expect to follow in achieving your goals.}

I have studied architecture maybe for almost one decade. During this period, I have received diplomas from a vocational high school to a master's degree in the architecture program.
Moreover, recently I have just found my interest on the urban planning. Therefore, I would like to continue specifically the urban planning in the future of my study.

The study of the city has gained much attention from many architecture departments in the Korean Universities. It is proven by the emergence of the variety programs that deal with city problems. One of the example is an architecture design studio graduate program. This program has covered urban planning issues such as the characters of architectural design which related to my proposal research.

In my research proposal, I would like to study about the identity of public spaces in Parepare, specifically in coastal areas. The urbanization that took place in Parepare lately has led to fast development. This development can cause the loss of local identity and the emergence of standardized and homogeneous city character. Finding a strong identity of public spaces is not only will attracts people in the city, but it will shows the good quality of environment.

The identity can be formed through the characters which lay inside a place. Indeed, public spaces have various characteristics. Those characteristics explain about what is in physical and social environments. By following architecture program in the Korean university, I would be able to finding characteristics of an architectural design that shape the identity of public spaces in Parepare.

%The result of this study would help policy maker, planner and architect in the city to determine their direction towrd the development waterfront public spaces. This study would also be foundation of planning in others architectural object inside a city, such square, sidewalk, and a monument.

\section{Future Plan after Study :Future plan in Korea or another country after finishing GKS program}
\section*{Please describe your future plans or goals after graduate studies.}

In short term, I might continue living in the South Korea for a while, if possible. The chance of living in the Korea would enable me to contribute more either in an academic, a professional, or a social way. For example, I would find job opportunities there where I would work with Korean citizen and involve in some Korean developments. Those will give me further exposures of the Korean culture, the education style, the working style and even the lifestyle that probably enhance my experience in Korea.

In long term, I would like to continue delivering a lecture in the university that has granted me further study consent. My lecture would have consisted of a mixed knowledge and experience from Korea and Indonesia. For example, I would like to present a comparison of the urban planning in the class specifically in public spaces between the city in Indonesia and Korea. As result, my future students would not only understand their country's architecture, but they would have wider perspective.

Beside delivering a lecture, I would probably doing some research. My research topics would discuss about the urban planning topic in the developing city. For example, I would explore the strength and weakness of public spaces on forming their identity between two countries. The result of a number research of mine could be utilized by the policy maker,planner, or even architect on figure out the problems by looking at the two cases examples.

It is encouraged that lecturer do community services. I have planned to involve in the community service programs with students. The job of a lecture is to supervise students while doing their program. This program will benefit a serviced place, usually it will be in remote area.

%----------------------------------------------------------------------------------------
%	BIBLIOGRAPHY
%----------------------------------------------------------------------------------------

%\bibliographystyle{apalike}

%\bibliography{biblio.bib}



\end{document}
